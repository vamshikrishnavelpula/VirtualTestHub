\documentclass{article}
\usepackage[utf8]{inputenc}
\usepackage{amsmath}

\title{Software Requirements Specification (SRS)\\ Virtual Test Hub (VTH)}
\author{}
\date{}

\begin{document}
	
	\maketitle
	
	\tableofcontents % This will generate the table of contents
	
	\newpage % Start a new page after the table of contents
	
	\section{Introduction}
	
	\subsection{Purpose}
	The purpose of this document is to specify the software requirements for the development of the Virtual Test Hub (VTH). This system aims to provide a robust platform for conducting online exams efficiently, securely, and at scale, addressing the challenges of traditional paper-based exams and current online systems.
	
	\subsection{Scope}
	The Virtual Test Hub will allow institutions to create, manage, and conduct exams online. Key features include user authentication, quiz management, multi-level quiz support, performance tracking, a repository of resources, and additional features like leaderboards and notifications. The system will primarily serve educational institutions seeking an alternative to physical exams.
	
	\subsection{Definitions, Acronyms, and Abbreviations}
	\begin{itemize}
		\item \textbf{VTH}: Virtual Test Hub
		\item \textbf{UG}: Undergraduate
		\item \textbf{PG}: Postgraduate
		\item \textbf{MERN}: MongoDB, Express.js, React, Node.js
	\end{itemize}
	
	\subsection{References}
	Project description document: ``VTH.txt''
	
	\section{Overall Description}
	
	\subsection{Product Perspective}
	The Virtual Test Hub is a web-based application that integrates various modules, including user authentication, quiz creation, performance tracking, and resource management, aimed at providing a seamless and secure environment for online exams.
	
	\subsection{Product Features}
	\begin{itemize}
		\item \textbf{User Authentication}: Login, sign-in, and registration.
		\item \textbf{Quiz Management}: Timer-based quizzes, question randomization, multiple question types, and progress indicators.
		\item \textbf{Multi-Level Quiz Support}: Separate quiz levels for Intermediate, Undergraduate, and Postgraduate.
		\item \textbf{Repository}: Books and articles for preparation, with search, filter, favorites, and download options.
		\item \textbf{User Profile}: Career details, activity logs, and performance reports.
		\item \textbf{Instructor Dashboard}: Allows educators to manage exams and view student performance.
		\item \textbf{Additional Features}: Leaderboards, notifications, reminders, and dark mode.
	\end{itemize}
	
	\subsection{User Classes and Characteristics}
	\begin{itemize}
		\item \textbf{Students}: Users who will take the quizzes and view their performance reports.
		\item \textbf{Instructors}: Users who will create, manage, and assess quizzes.
		\item \textbf{Administrators}: Users who manage system-wide settings and user roles.
	\end{itemize}
	
	\subsection{Operating Environment}
	The system will be web-based, compatible with modern web browsers, and hosted on a cloud platform to ensure scalability and reliability.
	
	\subsection{Design and Implementation Constraints}
	\begin{itemize}
		\item Must ensure system stability under high server load during exam sessions.
		\item Must implement security measures to prevent cheating and unauthorized access.
	\end{itemize}
	
	\subsection{Assumptions and Dependencies}
	\begin{itemize}
		\item The system will rely on a stable internet connection for both students and instructors.
		\item Users are expected to have basic computer literacy to interact with the system.
	\end{itemize}
	
	\section{Specific Requirements}
	
	\subsection{Functional Requirements}
	\begin{table}[h!]
		\centering
		\begin{tabular}{| m{2cm} | m{5cm} |}
			\hline
			\textbf{FR-ID} & \textbf{Description} \\
			\hline
			FR-1 & The system shall allow users to register, log in, and manage their profiles. Users should be able to update their personal information. \\
			\hline
			FR-2 & The system shall support quiz creation with multiple question types such as multiple choice, short answer, and essay. The system will also allow instructors to set time limits. \\
			\hline
			FR-3 & The system shall randomize quiz questions to prevent cheating. This feature ensures that no two users receive the same order of questions. \\
			\hline
			FR-4 & The system shall track time during quizzes and provide immediate feedback after submission. The feedback will include correct answers and explanations. \\
			\hline
			FR-5 & The system shall allow instructors to generate detailed performance reports, which include user activity, quiz scores, and progress over time. \\
			\hline
			FR-6 & The repository module shall allow students to search, filter, and download resources such as study materials, articles, and books. \\
			\hline
			FR-7 & The system shall provide leaderboard rankings based on quiz performance, and send notifications or reminders about upcoming exams or deadlines. \\
			\hline
		\end{tabular}
		\caption{Functional Requirements for Virtual Test Hub (VTH)}
	\end{table}
	
	\subsection{Non-Functional Requirements}
	\begin{itemize}
		\item \textbf{Performance}: The system must handle up to 1,000 concurrent users without crashing.
		\item \textbf{Security}: Must implement encryption for user data and exam content.
		\item \textbf{Usability}: The user interface should be intuitive and accessible to both students and instructors.
		\item \textbf{Reliability}: The system should have 99.9\% uptime during examination periods.
	\end{itemize}
	
	\subsection{User Interfaces}
	\begin{itemize}
		\item \textbf{Login Page}: Allows users to sign in or register.
		\item \textbf{Quiz Interface}: Displays questions, a timer, and progress indicators.
		\item \textbf{Instructor Dashboard}: Provides options to create quizzes, view reports, and manage users.
	\end{itemize}
	
	\section{System Models}
	
	\subsection{Use Case Diagram}
	\begin{itemize}
		\item \textbf{Actors}: Students, Instructors, Administrators
		\item \textbf{Use Cases}: Login, Quiz Creation, Quiz Attempt, View Results, Access Repository
	\end{itemize}
	
	\subsection{Data Flow Diagrams}
	\begin{itemize}
		\item \textbf{DFD Level 0}: User login, quiz creation, and result tracking.
		\item \textbf{DFD Level 1}: Detailed flow of data between user actions and system responses (e.g., fetching quiz questions, submitting results).
	\end{itemize}
	
	\section{External Interface Requirements}
	
	\subsection{User Interfaces}
	\begin{itemize}
		\item \textbf{Responsive Design}: The system must work on both desktop and mobile devices.
		\item \textbf{Theme Options}: Support for dark mode.
	\end{itemize}
	
	\subsection{Hardware Interfaces}
	The system will interact with standard web browsers on personal computers and mobile devices.
	
	\subsection{Software Interfaces}
	The system will use PHP and MySQL for server-side operations, with MERN stack support for the user interface.
	
	\section{Other Non-Functional Requirements}
	
	\subsection{Performance Requirements}
	Response times should be under 2 seconds for all user actions.
	
	\subsection{Security Requirements}
	The system must implement user session management and prevent unauthorized access using multi-factor authentication.
	
	\subsection{Maintainability}
	The system codebase will follow standard practices for maintainability and scalability.
	
\end{document}
